%! Author = melek
%! Date = 9.06.2022

% Preamble
\documentclass[11pt]{article}

% Packages
\usepackage{amsmath}
\DeclareMathOperator*{\argmax}{argmax}

\usepackage{graphicx}
\usepackage{amssymb}
\usepackage{bm}

\graphicspath{ {../images/} }


% Document
\begin{document}

    \maketitle
    \setcounter{section}{8}


    \section{Exercises}

    \subsection{Question}

    Show that tabular methods such as presented in Part I of this book are a special case of linear function approximation.
    What would the feature vectors be?

    \subsection*{Answer}

    In linear function approximation, the value function is product of feature(x) and weight(w) vectors.

    \noindention $ V(s) = x(s) W(s)$

    So to apply this to part 1, one may use x vector encoded as one hot in which only corresponding state is set to 1.
    Weight vector then should contain the state values.
    Product of a feature vector and weight vector produces the state value.


\end{document}


